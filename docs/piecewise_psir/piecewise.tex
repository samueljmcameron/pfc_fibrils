\documentclass[12pt]{article}
%\usepackage{helvet}
%\renewcommand{\familydefault}{\sfdefault}
\usepackage{amsfonts}
\usepackage{amsmath}
\usepackage{amssymb}
\usepackage{bm}
\usepackage{fullpage}
\usepackage{setspace}
\usepackage{graphicx}
\usepackage{gensymb}
\usepackage[nottoc,numbib]{tocbibind}
\usepackage{graphicx}
\usepackage{float}
\usepackage{braket}
\usepackage{titlesec}
\usepackage{siunitx}
\usepackage{mathtools}
\usepackage{tikz}
\usepackage[font={small}]{caption}
\usepackage{subcaption}
%\titlespacing*{\section}{0pt}{4pt}{4pt}
\usepackage[letterpaper, margin=2cm]{geometry}


\begin{document}
\pagenumbering{arabic}
\spacing{1.5}

%%%%%%%%%%%
% Begin Document
%%%%%%%%%%%

\section{Energy function for a piecewise linear twist angle field}
I will define the piecewise linear twist angle as
\begin{align}\label{eq:basicpiecewise}
\psi(r)=
\begin{cases}
	\psi_c^{\prime} r					& 0 \leq r \leq R_c \\
	\psi_s^{\prime} r +\psi_1			& R_c < r \leq R_s \\
	\psi_R^{\prime}r+\psi_2				& R_s < r \leq R.
\end{cases}
\end{align}
which have two constraints,
\begin{subequations}
\begin{align}
&\psi_1=(\psi_c^{\prime}-\psi_s^{\prime})R_c\label{eq:psi1},\\
&\psi_2=(\psi_s^{\prime}-\psi_R^{\prime})R_s+(\psi_c^{\prime}-\psi_s^{\prime})R_c\label{eq:psi2},
\end{align}
\end{subequations}
to ensure continuity of $\psi(r)$.

Next, I will insert this into the free energy per unit volume,
\begin{align}\label{eq:startE}
E(R,\eta,\delta;\psi(r))&=\frac{2}{R^2}\int_0^{R}rdr\left[\frac{1}{2}\left(\psi'+\frac{\sin2\psi}{2r}-1\right)^2+\frac{1}{2}K_{33}\frac{\sin^4\psi}{r^2}\right]\nonumber\\
&\phantom{=}+\frac{\Lambda\delta^2}{2R^2}\int_0^{R}rdr\left(\frac{4\pi^2}{\cos^2\psi}-\eta^2\right)^2+\frac{\omega\delta^2}{2}\left(\frac{3}{4}\delta^2-1\right)\nonumber\\
&\phantom{=}-\frac{(1+k_{24})}{R^2}\sin\psi(R)+\frac{2\gamma}{R}.
\end{align}

The resulting equation can be written as
\begin{align}\label{eq:piecewise_E}
E(R,\eta,&\delta,R_c,R_s,\psi_c^{\prime},\psi_s^{\prime},\psi_R^{\prime})=\nonumber\\
&\frac{2}{R^2}\left(\frac{1}{4}(u(0,R_c,\psi_c^{\prime})+u(R_c,R_s,\psi_s^{\prime})+u(R_s,R,\psi_R^{\prime}))+\frac{1}{8}(f_1(0,R_c,0,\psi_c^{\prime})\right.\nonumber\\
&+f_1(R_c,R_s,\psi_1,\psi_s^{\prime})+f_1(R_s,R,\psi_2,\psi_R^{\prime}))+\frac{1}{2}K_{33}(f_2(0,R_c,0,\psi_c^{\prime})+f_2(R_c,R_s,\psi_1,\psi_s^{\prime})\nonumber\\
&\left.+f_2(R_s,R,\psi_2,\psi_R^{\prime}))+\frac{1}{4}(v(0,R_c,0,\psi_c^{\prime})+v(R_c,R_s,\psi_1,\psi_s^{\prime})+v(R_s,R,\psi_2,\psi_R^{\prime}))\right)\nonumber\\
&+\frac{\Lambda\delta^2}{2R^2}\left(\vphantom{\frac{1}{2}}16\pi^4(g_2(0,R_c,0,\psi_c^{\prime})+g_2(R_c,R_s,\psi_1,\psi_s^{\prime})+g_2(R_s,R,\psi_2,\psi_R^{\prime}))\right.\nonumber\\
&\left.-8\pi^2\eta^2(g_1(0,R_c,0,\psi_c^{\prime})+g_1(R_c,R_s,\psi_1,\psi_s^{\prime})+g_1(R_s,R,\psi_2,\psi_R^{\prime}))+\frac{\eta^4}{2}R^2\right)\nonumber\\
&+\frac{\omega\delta^2}{2}\left(\frac{3}{4}\delta^2-1\right)-\frac{(1+k_{24})}{R^2}\sin(\psi_R^{\prime}R+\psi_2)+\frac{2\gamma}{R}
\end{align}

which utilizes a derivation shown in the appendix (see eqns \ref{eq:first_int} and \ref{eq:second_int}, as well as the definitions of the functions $u$, $v$, $f_{\alpha}$, and $g_{\alpha}$ in equations \ref{eq:ufunc}, \ref{eq:vfunc}, \ref{eq:falpha}, and \ref{eq:galpha}, respectively). I will minimize this equation subject to the constraint equations \ref{eq:psi1} and \ref{eq:psi2} to determine the equilibrium configuration of the fibril.

\section{Differentiation of the energy}
The derivative with respect to the fibril radius is
\begin{align}
\frac{\partial E}{\partial R}=&\nonumber\\
&-\frac{4}{R^3}\left(\frac{1}{4}(u(0,R_c,\psi_c^{\prime})+u(R_c,R_s,\psi_s^{\prime})+u(R_s,R,\psi_R^{\prime}))+\frac{1}{8}(f_1(0,R_c,0,\psi_c^{\prime})\right.\nonumber\\
&+f_1(R_c,R_s,\psi_1,\psi_s^{\prime})+f_1(R_s,R,\psi_2,\psi_R^{\prime}))+\frac{1}{2}K_{33}(f_2(0,R_c,0,\psi_c^{\prime})+f_2(R_c,R_s,\psi_1,\psi_s^{\prime})\nonumber\\
&\left.+f_2(R_s,R,\psi_2,\psi_R^{\prime}))+\frac{1}{4}(v(0,R_c,0,\psi_c^{\prime})+v(R_c,R_s,\psi_1,\psi_s^{\prime})+v(R_s,R,\psi_2,\psi_R^{\prime}))\right)\nonumber\\
&+\frac{2}{R^2}\left(\frac{1}{4}\left.\frac{\partial u(R_s,x_2,\psi_R^{\prime})}{\partial x_2}\right|_{x_2=R}+\frac{1}{8}\left.\frac{\partial f_1(R_s,x_2,\psi_2,\psi_R^{\prime})}{\partial x_2}\right|_{x_2=R}+\frac{1}{2}K_{33}\left.\frac{\partial f_2(R_s,x_2,\psi_2,\psi_R^{\prime})}{\partial x_2}\right|_{x_2=R}\right.\nonumber\\
&\left.+\frac{1}{4}\left.\frac{\partial v(R_s,x_2,\psi_2,\psi_R^{\prime})}{\partial x_2}\right|_{x_2=R}\right)\nonumber\\
&-\frac{\Lambda\delta^2}{R^3}\left(\vphantom{\frac{1}{2}}16\pi^4(g_2(0,R_c,0,\psi_c^{\prime})+g_2(R_c,R_s,\psi_1,\psi_s^{\prime})+g_2(R_s,R,\psi_2,\psi_R^{\prime}))\right.\nonumber\\
&\left.-8\pi^2\eta^2(g_1(0,R_c,0,\psi_c^{\prime})+g_1(R_c,R_s,\psi_1,\psi_s^{\prime})+g_1(R_s,R,\psi_2,\psi_R^{\prime}))+\frac{\eta^4}{2}R^2\right)\nonumber\\
&+\frac{\Lambda\delta^2}{2R^2}\left(16\pi^4\left.\frac{\partial g_2(R_s,x_2,\psi_2,\psi_R^{\prime})}{\partial x_2}\right|_{x_2=R}-8\pi^2\eta^2\left.\frac{\partial g_1(R_s,x_2,\psi_2,\psi_R^{\prime})}{\partial x_2}\right|_{x_2=R}+\eta^4R\right)\nonumber\\
&+\frac{2(1+k_{24})}{R^3}\sin(\psi_R^{\prime}R+\psi_2)-\frac{\psi_R^{\prime}(1+k_{24})}{R^2}\cos(\psi_R^{\prime}R+\psi_2)-\frac{2\gamma}{R^2}.\label{eq:dEdR}
\end{align}
The derivative with respect to the inverse period of density modulations, $\eta$, is
\begin{align}
\frac{\partial E}{\partial \eta}=&
\frac{\Lambda\delta^2}{2R^2}\left(-16\pi^2\eta(g_1(0,R_c,0,\psi_c^{\prime})+g_1(R_c,R_s,\psi_1,\psi_s^{\prime})+g_1(R_s,R,\psi_2,\psi_R^{\prime}))+2\eta^3R^2\right)\label{eq:dEdeta}
\end{align}
The derivative with respect to the size of the density modulations, $\delta$, is
\begin{align}
\frac{\partial E}{\partial \delta}=&\frac{\Lambda\delta}{R^2}\left(\vphantom{\frac{1}{2}}16\pi^4(g_2(0,R_c,0,\psi_c^{\prime})+g_2(R_c,R_s,\psi_1,\psi_s^{\prime})+g_2(R_s,R,\psi_2,\psi_R^{\prime}))\right.\nonumber\\
&\left.-8\pi^2\eta^2(g_1(0,R_c,0,\psi_c^{\prime})+g_1(R_c,R_s,\psi_1,\psi_s^{\prime})+g_1(R_s,R,\psi_2,\psi_R^{\prime}))+\frac{\eta^4}{2}R^2\right)+\omega\delta\left(\frac{3}{2}\delta^2-1\right).
\end{align}


\section{Detailed calculations}
For a general linear function of the form $\psi(r)=\psi_{ab}^{\prime}r+\psi_0$ in the region $a<r<b$, the two integrals in eqn \ref{eq:startE} become
\begin{align}\label{eq:first_int}
\int_{a}^{b}&rdr\left[\frac{1}{2}\left(\psi_{ab}^{\prime}+\frac{\sin(2(\psi_{ab}^{\prime}r+\psi_0))}{2r}-1\right)^2+\frac{1}{2}K_{33}\frac{\sin^4(\psi_{ab}^{\prime}r+\psi_0)}{r^2}\right]\nonumber\\
&=\int_{a}^{b}dr\left(\frac{(1-\psi_{ab}^{\prime})^2}{2}r+\frac{1}{8}\frac{\sin^2(2(\psi_{ab}^{\prime}r+\psi_0))}{r}-\frac{(1-\psi_{ab}^{\prime})}{2}\sin(2(\psi_{ab}^{\prime}r+\psi_0))+\frac{1}{2}K_{33}\frac{\sin^4(\psi_{ab}^{\prime}r+\psi_0)}{r}\right)\nonumber\\
&=\left(\frac{1}{4}u(a,b,\psi_{ab}^{\prime})+\frac{1}{8}f_1(a,b,\psi_0,\psi_{ab}^{\prime})+\frac{1}{2}K_{33}f_2(a,b,\psi_0,\psi_{ab}^{\prime})+\frac{1}{4}v(a,b,\psi_0,\psi_{ab}^{\prime})\right)
\end{align}
and
\begin{align}\label{eq:second_int}
\int_{a}^{b}rdr\bigg(\frac{4\pi^2}{\cos^2(\psi_{ab}^{\prime}r+\psi_0)}&-\eta^2\bigg)^2\nonumber\\
=&\int_{a}^{b}dr\left(\frac{16\pi^4r}{\cos^4(\psi_{ab}^{\prime}r+\psi_0)}-\frac{8\pi^2r}{\cos^2(\psi_{ab}^{\prime}r+\psi_0)}\eta^2+\eta^4r\right)\nonumber\\
=&\left(16\pi^4g_2(a,b,\psi_0,\psi_{ab}^{\prime})-8\pi^2\eta^2g_1(a,b,\psi_0,\psi_{ab}^{\prime})+\frac{\eta^4}{2}(b^2-a^2)\right)
\end{align}
where I have defined the functions
\begin{subequations}
\begin{align}
&u(x_1,x_2,\zeta)=(1-\zeta)^2(x_2^2-x_1^2),\label{eq:ufunc}\\
&v(x_1,x_2,\xi,\zeta)=\frac{(1-\zeta)}{\zeta}(\cos(2(\zeta x_2+\xi))-\cos(2(\zeta x_1+\xi))),\label{eq:vfunc}\\
&f_{\alpha}(x_1,x_2,\xi,\zeta)=\int_{x_1}^{x_2}du\frac{\sin^{2\alpha}\left(\frac{2}{\alpha}(\zeta u+\xi)\right)}{u},\label{eq:falpha}\\
&g_{\alpha}(x_1,x_2,\xi,\zeta)=\int_{x_1}^{x_2}du\frac{u}{\cos^{2\alpha}(\zeta u+\xi)}.\label{eq:galpha}
\end{align}
\end{subequations}
For $\zeta\ll1$, I can expand the final three of these equations up to $\mathcal{O}(\zeta^4)$ using trigonometric identities to get
\begin{subequations}
\begin{align}
&v(x_1,x_2,\xi,\zeta)= -2(1-\zeta)\sin(2\xi)(x_2-x_1)-2(1-\zeta)\cos(2\xi)(x_2^2-x_1^2)\zeta\nonumber\\
&\phantom{v(x_1,x_2,\xi,\zeta)= }+\frac{4}{3}(1-\zeta)\sin(2\xi)(x_2^3-x_1^3)\zeta^2+\frac{2}{3}\cos(2\xi)(x_2^4-x_1^4)\zeta^3,\\
&f_1(x_1,x_2,\xi,\zeta)=\sin^2\left(2\xi\right)\ln\frac{x_2}{x_1}+4\zeta(x_2-x_1)\cos(2\xi)\sin(2\xi)\nonumber\\
&\phantom{f_1(x_1,x_2,\xi,\zeta)=}+2\zeta^2(x_2^2-x_1^2)\left(\cos^2(2\xi)-\sin^2(2\xi)\right)-\frac{32}{9}\zeta^3(x_2^3-x_1^3)\sin(2\xi)\cos(2\xi)\\
&f_2(x_1,x_2,\xi,\zeta)=\sin^4\xi\ln\frac{x_2}{x_1}+4\zeta(x_2-x_1)\sin^3\xi\cos\xi+\zeta^2(x_2^2-x_1^2)\sin^2\xi(\cos^2\xi-\sin^2\xi)\nonumber\\
&\phantom{f_2(x_1,x_2,\xi,\zeta)=}+\frac{4}{3}\zeta^3(x_2^3-x_1^3)\sin\xi\cos\xi\left(\cos^2\xi-5\sin^2\xi\right)\\
&g_1(x_1,x_2,\xi,\zeta)=\frac{1}{\cos^2\xi}\left(\frac{x_2^2-x_1^2}{2}+\frac{2\zeta(x_2^3-x_1^2)}{3}\tan\xi+\frac{\zeta^2(x_2^4-x_1^4)(3\tan^2\xi+1)}{4}\right.\nonumber\\
&\phantom{g_1(x_1,x_2,\xi,\zeta)=}\left.+\frac{4\zeta^3(x_2^5-x_1^5)(4+3\tan^2\xi)\tan\xi}{15}\right)\\
&g_2(x_1,x_2,\xi,\zeta)=\frac{1}{\cos^4\xi}\left(\frac{x_2^2-x_1^2}{2}+\frac{4\zeta(x_2^3-x_1^3)\tan\xi}{3}+\frac{\zeta^2(x_2^4-x_1^4)(1+5\tan^2\xi)}{2}\right.\nonumber\\
&\phantom{g_2(x_1,x_2,\xi,\zeta)=}\left.+\frac{\zeta^3(x_2^5-x_1^5)(60\tan^2\xi+28)\tan\xi}{15}\right)
\end{align}
\end{subequations}

The derivatives of these functions are listed below:
\begin{subequations}
\begin{align}
&\frac{\partial u}{\partial x_1}=-2(1-\zeta)^2x_1\label{eq:dudx_1}\\
&\frac{\partial u}{\partial x_2}=2(1-\zeta)^2x_2\label{eq:dudx_2}\\
&\frac{\partial u}{\partial \xi} = 0\label{eq:dudxi}\\
&\frac{\partial u}{\partial\zeta}=-2\zeta(1-\zeta)(x_2^2-x_1^2)\label{eq:dudzeta}
\end{align}
\end{subequations}
\begin{subequations}
\begin{align}
&\frac{\partial v}{\partial x_1} = 2(1-\zeta)\sin(2(\zeta x_1+\xi))\label{eq:dvdx_1}\\
&\frac{\partial v}{\partial x_2} = -2(1-\zeta)\sin(2(\zeta x_1+\xi))\label{eq:dvdx_2}\\
&\frac{\partial v}{\partial \xi} = 
	\begin{cases}
	-4\cos(2\xi)(x_2-x_1)+(4\cos(2\xi)(x_2-x_1)+4\sin(2\xi)(x_2^3-x_1^3))\zeta,	&\zeta=0\\
	-2\frac{(1-\zeta)}{\zeta}(\sin(2(\zeta x_2+\xi))-\sin(2(\zeta x_1+\xi))),			&\zeta\neq0
	\end{cases}\label{eq:dvdxi}\\
&\frac{\partial v}{\partial \zeta} = 
	\begin{cases}
	2\sin(2\xi)(x_2-x_1)-2\cos(2\xi)(x_2^2-x_1^2)+4(\cos(2\xi)(x_2^2-x_1^2)+\frac{2}{3}\sin(2\xi)(x_2^3-x_1^3))\zeta,	& \zeta=0\\
	\frac{-2(1-\zeta)}{\zeta}(x_2\sin(2(\zeta x_2+\xi))-x_1\sin(2(\zeta x_1+\xi)))-\frac{1}{\zeta^2}(\cos(2(\zeta x_2+\xi))-\cos(2(\zeta x_1 +\xi))),	& \zeta\neq0
	\end{cases}\label{eq:dvdzeta}
\end{align}
\end{subequations}
\begin{subequations}
\begin{align}
&\frac{\partial f_{\alpha}}{\partial x_1} =
	\begin{cases}
	\infty,													& x_1=0, \xi\neq0\\
	-\left(\frac{2\zeta}{\alpha}\right)^{2\alpha}x_1^{2\alpha-1},			& x_1=0, \xi=0\\
	-\frac{\sin^{2\alpha}\left(\frac{2}{\alpha}(\zeta x_1+\xi)\right)}{x_1},	& x_1\neq0
	\end{cases}\label{eq:dfdx_1}\\
&\frac{\partial f_{\alpha}}{\partial x_2} =
	\begin{cases}
	\infty,													& x_1=0, \xi\neq0\\
	\left(\frac{2\zeta}{\alpha}\right)^{2\alpha}x_2^{2\alpha-1},			& x_1=0, \xi=0\\
	\frac{\sin^{2\alpha}\left(\frac{2}{\alpha}(\zeta x_1+\xi)\right)}{x_1},	& x_1\neq0
	\end{cases}\label{eq:dfdx_2}\\
&\frac{\partial f_{\alpha}}{\partial \xi} = 
	\begin{cases}
	\infty,													& x_1 = 0,\xi\neq0\\
	\int_{x_1}^{x_2}du\frac{4\sin\left(\frac{2}{\alpha}(\zeta u+\xi)\right)\cos\left(\frac{2}{\alpha}(\zeta u+\xi)\right)}{u},	& x_1\neq0
	\end{cases}\label{eq:dfdxi}\\
&\frac{\partial f_{\alpha}}{\partial\zeta} = 
	\begin{cases}
	4(x_2-x_1)\cos(2\xi)\sin(2\xi)+4(x_2^2-x_1^2)(\cos^2(2\xi)-\sin^2(2\xi))\zeta,	&\zeta=0,\alpha=1\\
	4(x_2-x_1)\sin^3(\xi)\cos(\xi)+2(x_2^2-x_1^2)\sin^2(\xi)(3\cos^2(\xi)-\sin^2(\xi))\zeta,	&\zeta=0,\alpha=2\\
	\frac{1}{4\zeta}\left(\sin\left(\frac{2}{\alpha}(\zeta x_2+\xi)\right)-\sin\left(\frac{2}{\alpha}(\zeta x_1+\xi)\right)\right),		&\zeta\neq0
	\end{cases}\label{eq:dfdzeta}
\end{align}
\end{subequations}
\begin{subequations}
\begin{align}
&\frac{\partial g_{\alpha}}{\partial x_1}=\frac{-x_1}{\cos^{2\alpha}(\zeta x_1 +\xi)}\label{eq:dgdx_1}\\
&\frac{\partial g_{\alpha}}{\partial x_2}=\frac{x_2}{\cos^{2\alpha}(\zeta x_2+\xi)}\label{eq:dgdx_2}\\
&\frac{\partial g_{\alpha}}{\partial\xi}=\int_{x_1}^{x_2}du\frac{2\alpha u \sin(\zeta u+\xi)}{\cos^{2\alpha+1}(\zeta u+\xi)}\label{eq:dgdxi}\\
&\frac{\partial g_{\alpha}}{\partial\zeta}=\int_{x_1}^{x_2}du\frac{2\alpha u^2\sin(\zeta u+\xi)}{\cos^{2\alpha+1}(\zeta u +\xi)}\label{eq:dgdzeta}
\end{align}
\end{subequations}

	
%%%%%%%%%%%%%%%
% Bibliography
%%%%%%%%%%%%%%%
\clearpage
\bibliography{pfc}
\bibliographystyle{unsrt}

\end{document}

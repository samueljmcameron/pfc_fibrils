\documentclass[12pt]{article}
%\usepackage{helvet}
%\renewcommand{\familydefault}{\sfdefault}
\usepackage{amsfonts}
\usepackage{amsmath}
\usepackage{amssymb}
\usepackage{bm}
\usepackage{fullpage}
\usepackage{setspace}
\usepackage{graphicx}
\usepackage{gensymb}
\usepackage[nottoc,numbib]{tocbibind}
\usepackage{graphicx}
\usepackage{float}
\usepackage{braket}
\usepackage{titlesec}
\usepackage{siunitx}
\usepackage{mathtools}
\usepackage{tikz}
\usepackage[font={small}]{caption}
\usepackage{subcaption}
%\titlespacing*{\section}{0pt}{4pt}{4pt}
\usepackage[letterpaper, margin=2cm]{geometry}


\begin{document}
\pagenumbering{arabic}
\spacing{1.5}

%%%%%%%%%%%
% Begin Document
%%%%%%%%%%%

\section{Piecewise linear definition}
I will define the piecewise linear twist angle as
\begin{align}\label{eq:basicpiecewise}
\psi(r)=
\begin{cases}
	\psi_c^{\prime} r															& 0 \leq r \leq R_c \\
	\psi_s^{\prime} r + (\psi_c^{\prime}-\psi_s^{\prime})R_c								& R_c < r \leq R_s \\
	\psi_R^{\prime}r+(\psi_s^{\prime}-\psi_R^{\prime})R_s+(\psi_c^{\prime}-\psi_s^{\prime})R_c	& R_s < r \leq R.
\end{cases}
\end{align}

Next, I will insert this into the free energy per unit volume,
\begin{align}\label{eq:startE}
E(R,\eta,\delta;\psi(r))&=\frac{2}{R^2}\int_0^{R}rdr\left[\frac{1}{2}\left(\psi'+\frac{\sin2\psi}{2r}-1\right)^2+\frac{1}{2}K_{33}\frac{\sin^4\psi}{r^2}\right]\nonumber\\
&\phantom{=}+\frac{\Lambda\delta^2}{2R^2}\int_0^{R}rdr\left(\frac{4\pi^2}{\cos^2\psi}-\eta^2\right)^2\nonumber\\
&\phantom{=}+\frac{\omega\delta^2}{2}\left(\frac{3}{4}\delta^2-1\right)-\frac{(1+k_{24})}{R^2}\sin\psi(R)+\frac{2\gamma}{R}.
\end{align}

Inserting the form of $\psi(r)$ from eqn \ref{eq:basicpiecewise} I get the free energy as function of 8 variables,
\begin{align}
E(R,\eta,\delta,&R_c,R_s,\psi_c^{\prime},\psi_s^{\prime},\psi_R^{\prime}) = \frac{2}{R^2}\left[q(\eta,\delta,0,0,R_c,0,0,\psi_c^{\prime})+q(\eta,\delta,0,R_c,R_s,0,\psi_c^{\prime},\psi_s^{\prime})+q(\eta,\delta,R_c,R_s,R,\psi_c^{\prime},\psi_s^{\prime},\psi_R^{\prime})\right]
\end{align}

I will look first only at the two integral terms in eqn \ref{eq:startE}, as that is where the piecewise linear function enters the calcution. Starting with the region $0\leq r \leq R_c$, I get
\begin{align}
\phantom{=}&\frac{2}{R^2}\int_0^{R_c}rdr\left[\frac{1}{2}\left(\psi_0^{\prime}+\frac{\sin(2\psi_0^{\prime}r)}{2r}-1\right)^2+\frac{1}{2}K_{33}\frac{\sin^4(\psi_0^{\prime}r)}{r^2}\right]+\frac{\Lambda\delta^2}{2R^2}\int_0^{R_c}rdr\left(\frac{4\pi^2}{\cos^2(\psi_0^{\prime}r)}-\eta^2\right)^2\nonumber\\
=&\frac{2}{R^2}\left(\frac{(\psi_0^{\prime}-1)^2}{4}R_c^2+\frac{\psi_0^{\prime}}{4}f_1(2\psi_0^{\prime}R_c)+\frac{(\psi_0^{\prime}-1)}{4\psi_0^{\prime}}(1-\cos(2\psi_0^{\prime}R_c))+\frac{\psi_0^{\prime}}{2}K_{33}f_2(\psi_0^{\prime}R_c)\right)\nonumber\\
\phantom{=}&+\frac{\Lambda\delta^2}{2R^2}\left(\frac{16\pi^4}{\psi_0^{\prime}}g_2(\psi_0^{\prime}R_c)-\frac{8\pi^2\eta^2}{\psi_0^{\prime}}g_1(\psi_0^{\prime}R_c)+\frac{\eta^4}{2}R_c^2\right)\nonumber\\
\end{align}

Next, looking at the shelf region $R_c\leq r < R_s$, I get
\begin{align}
\end{align}


Finally, in the outer region $R_s\leq r < R$, I get
\begin{align}
\end{align}

Adding all of this together, I end up with the free energy per unit volume of the d-banded fibril which is a function of 8 variables,
\begin{align}
E(R,\eta,\delta,&R_c,R_s,\psi_0^{\prime},\psi_c,\psi_R^{\prime})\nonumber\\
=&\frac{2}{R^2}\left(\frac{(\psi_0^{\prime}-1)^2}{4}R_c^2+\frac{\psi_0^{\prime}}{4}f_1(2\psi_0^{\prime}R_c)+\frac{(\psi_0^{\prime}-1)}{4\psi_0^{\prime}}(1-\cos(2\psi_0^{\prime}R_c))+\frac{\psi_0^{\prime}}{2}K_{33}f_2(\psi_0^{\prime}R_c)\right)\nonumber\\
\phantom{=}&+\frac{\Lambda\delta^2}{2R^2}\left(\frac{16\pi^4}{\psi_0^{\prime}}g_2(\psi_0^{\prime}R_c)-\frac{8\pi^2\eta^2}{\psi_0^{\prime}}g_1(\psi_0^{\prime}R_c)+\frac{\eta^4}{2}R_c^2\right)\nonumber\\
\phantom{=}&\frac{\sin^2(2\psi_c)}{4R^2}\ln\left(\frac{R_s}{R_c}\right)-\frac{\sin(2\psi_c)}{R^2}(R_s-R_c)+\frac{R_s^2-R_c^2}{2R^2}\left(1+\frac{\Lambda\delta^2}{2}\left(\frac{4\pi^2}{\cos^2\psi_c}-\eta^2\right)^2\right)\nonumber\\
=&\frac{2}{R^2}\left(\frac{(\psi_R^{\prime}-1)^2}{4}(R^2-R_s^2)+\frac{\psi_R^{\prime}}{4}(f_1(2\psi_R^{\prime}R)-f_1(2\psi_R^{\prime}R_s))+\frac{(\psi_R^{\prime}-1)}{4\psi_R^{\prime}}(\cos(2\psi_R^{\prime}R_s)-\cos(2\psi_R^{\prime}R))\right.\nonumber\\
\phantom{=}&\left.+\frac{\psi_R^{\prime}}{2}K_{33}(f_2(\psi_R^{\prime}R)-f_2(\psi_R^{\prime}R))\right)\nonumber\\
\phantom{=}&+\frac{\Lambda\delta^2}{2R^2}\left(\frac{16\pi^4}{\psi_R^{\prime}}(g_2(\psi_R^{\prime}R)-g_2(\psi_R^{\prime}R_s))-\frac{8\pi^2\eta^2}{\psi_R^{\prime}}(g_1(\psi_R^{\prime}R)-g_1(\psi_R^{\prime}R_s))+\frac{\eta^4}{2}(R^2-R_s^2)\right)
\end{align}




\section{Detailed calculations}
For a general linear function of the form $\psi(r)=\psi_{ab}^{\prime}r+\psi_0$ in the region $a<r<b$, the two integrals in eqn \ref{eq:startE} become
\begin{align}
\int_{a}^{b}&rdr\left[\frac{1}{2}\left(\psi_{ab}^{\prime}+\frac{\sin(2(\psi_{ab}^{\prime}r+\psi_0))}{2r}-1\right)^2+\frac{1}{2}K_{33}\frac{\sin^4(\psi_{ab}^{\prime}r+\psi_0)}{r^2}\right]\nonumber\\
&=\int_{a}^{b}dr\left(\frac{(1-\psi_{ab}^{\prime})^2}{2}r+\frac{1}{8}\frac{\sin^2(2(\psi_{ab}^{\prime}r+\psi_0))}{r}-\frac{(1-\psi_{ab}^{\prime})}{2}\sin(2(\psi_{ab}^{\prime}r+\psi_0))+\frac{1}{2}K_{33}\frac{\sin^4(\psi_{ab}^{\prime}r+\psi_0)}{r}\right)\nonumber\\
&=\left(\frac{1}{4}u(a,b,\psi_{ab}^{\prime})+\frac{1}{8}f_1(a,b,\psi_0,\psi_{ab}^{\prime})+\frac{1}{2}K_{33}f_2(a,b,\psi_0,\psi_{ab}^{\prime})+\frac{1}{4}v(a,b,\psi_0,\psi_{ab}^{\prime})\right)
\end{align}
and
\begin{align}
\int_{a}^{b}rdr\bigg(\frac{4\pi^2}{\cos^2(\psi_{ab}^{\prime}r+\psi_0)}&-\eta^2\bigg)^2\nonumber\\
=&\int_{a}^{b}dr\left(\frac{16\pi^4r}{\cos^4(\psi_{ab}^{\prime}r+\psi_0)}-\frac{8\pi^2r}{\cos^2(\psi_{ab}^{\prime}r+\psi_0)}\eta^2+\eta^4r\right)\nonumber\\
=&\left(16\pi^4g_2(a,b,\psi_0,\psi_{ab}^{\prime})-8\pi^2\eta^2g_1(a,b,\psi_0,\psi_{ab}^{\prime})+\frac{\eta^4}{2}(b^2-a^2)\right)
\end{align}
where I have defined the functions
\begin{subequations}
\begin{align}
&u(x_1,x_2,\zeta)=(1-\zeta)^2(x_2^2-x_1^2),\label{eq:ufunc}\\
&v(x_1,x_2,\xi,\zeta)=\frac{(1-\zeta)}{\zeta}(\cos(2(\zeta x_2+\xi))-\cos(2(\zeta x_1+\xi))),\label{eq:vfunc}\\
&f_{\alpha}(x_1,x_2,\xi,\zeta)=\int_{x_1}^{x_2}du\frac{sin^{2\alpha}\left(\frac{2}{\alpha}(\zeta u+\xi)\right)}{u},\label{eq:falpha}\\
&g_{\alpha}(x_1,x_2,\xi,\zeta)=\int_{x_1}^{x_2}du\frac{u}{cos^{2\alpha}(\zeta u+\xi)}.\label{eq:galpha}\\
\end{align}
\end{subequations}
For $\zeta\ll1$, I can expand the final three of these equations up to $\mathcal{O}(\zeta^4)$ using trigonometric identities to get
\begin{subequations}
\begin{align}
&v(x_1,x_2,\xi,\zeta)= -2(1-\zeta)\sin(2\xi)(x_2-x_1)-2(1-\zeta)\cos(2\xi)(x_2^2-x_1^2)\zeta\nonumber\\
&\phantom{v(x_1,x_2,\xi,\zeta)= }+\frac{4}{3}(1-\zeta)\sin(2\xi)(x_2^3-x_1^3)\zeta^2+\frac{2}{3}\cos(2\xi)(x_2^4-x_1^4)\zeta^3,\\
&f_1(x_1,x_2,\xi,\zeta)=\sin^2\left(\frac{2\xi}{\alpha}\right)\ln\frac{x_2}{x_1}+4\zeta(x_2-x_1)\cos(2\xi)\sin(2\xi)\nonumber\\
&\phantom{f_1(x_1,x_2,\xi,\zeta)=}+2\zeta^2(x_2^2-x_1^2)\left(\cos^2(2\xi)-\sin^2(2\xi)\right)-\frac{32}{9}\zeta^3(x_2^3-x_1^3)\sin(2\xi)\cos(2\xi)\\
&f_2(x_1,x_2,\xi,\zeta)=\sin^4\xi\ln\frac{x_2}{x_1}+4\zeta(x_2-x_1)\sin^3\xi\cos\xi+\zeta^2(x_2^2-x_1^2)\sin^2\xi(\cos^2\xi-\sin^2\xi)\nonumber\\
&\phantom{f_2(x_1,x_2,\xi,\zeta)=}+\frac{4}{3}\zeta^3(x_2^3-x_1^3)\sin\xi\cos\xi\left(\cos^2\xi-5\sin^2\xi\right)
\end{align}
\end{subequations}

%%%%%%%%%%%%%%%
% Bibliography
%%%%%%%%%%%%%%%
\clearpage
\bibliography{pfc}
\bibliographystyle{unsrt}

\end{document}
